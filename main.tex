\documentclass{article}
\usepackage{graphicx} % Required for inserting images

\title{test}
\author{batunasun }
\date{January 2025}

\begin{document}

\maketitle

\section{Introduction}
- |H| = 8

- |Z| = 2

- |X| = 15 (conjugates of Z)

- |Y| = 15 (conjugates of H)

- |C| = 8 (centralizer of Z)

$\Delta $


Given the symmetric group \( S_5 \), we need to find the number of elements in the subgroups \( H \), \( Z \), \( X \), \( Y \), and \( C \).

1. **Subgroup \( H \)**:
   - Generated by the permutations (1 2) and (1 3 2 4).
   - These generators form a dihedral group \( D_4 \) of order 8.
   - Therefore, \( |H| = 8 \).

2. **Center \( Z \) of \( H \)**:
   - The center of \( D_4 \) is the subgroup generated by the element (1 2)(3 4).
   - This subgroup has order 2.
   - Therefore, \( |Z| = 2 \).

3. **Set \( X \)**:
   - \( X \) is the set of conjugates of \( Z \) in \( S_5 \).
   - The non-identity element of \( Z \) is a double transposition, and there are 15 double transpositions in \( S_5 \).
   - Each double transposition generates a conjugate subgroup of \( Z \).
   - Therefore, \( |X| = 15 \).

4. **Set \( Y \)**:
   - \( Y \) is the set of conjugates of \( H \) in \( S_5 \).
   - \( H \) is a Sylow 2-subgroup of \( S_5 \), and the number of Sylow 2-subgroups in \( S_5 \) is 15.
   - Therefore, \( |Y| = 15 \).

5. **Centralizer \( C \)**:
   - \( C \) is the centralizer of \( Z \) in \( S_5 \).
   - The centralizer of the double transposition (1 2)(3 4) in \( S_5 \) is isomorphic to the dihedral group \( D_4 \), which has order 8.
   - Therefore, \( |C| = 8 \).

### Final Answer
\(|H| = \boxed{8}\), \(|Z| = \boxed{2}\), \(|X| = \boxed{15}\), \(|Y| = \boxed{15}\), \(|C| = \boxed{8}\)
\end{document}
